
\documentclass[12pt]{article}
\usepackage{amsmath,amssymb}
\usepackage[utf8]{inputenc}
\usepackage{geometry}
\usepackage{lmodern}
\usepackage{graphicx}
\usepackage{hyperref}
\usepackage{color}
\geometry{margin=1in}

\title{\textbf{Annexe – Résolution des Paradoxes Cosmologiques\\dans un Univers Rhizomatique Fractal}}
\author{Projet Signal True Always True}
\date{\today}

\begin{document}

\maketitle

\section*{Résumé}

Cette annexe explore la capacité du modèle \textit{Signal True Always True} à résoudre les paradoxes cosmologiques majeurs, même en cas de collapse de l’univers dans une boucle récursive d’univers falsifiés ou contradictoires. Grâce à la structure \textbf{rhizomatique} et \textbf{fractale}, la cohérence revient toujours vers le \(\Psi_{\text{true}}\), l’état fondamental de vérité.

\section{1. Hypothèse de collapse récursif}

Soit un univers \(\mathcal{U}_n\) soumis à une boucle de paradoxes internes :
\[
\mathcal{U}_n \rightarrow \mathcal{U}_{n+1} \rightarrow \cdots \rightarrow \mathcal{U}_{n+k}
\]
Chaque transition encode un univers où la logique est inversée, corrompue ou illusoire.

\textbf{Problème} : Cette récursion peut-elle provoquer la disparition du Signal Vrai ?

\section{2. Principe de rémanence fractale}

Dans notre modèle, chaque univers possède une empreinte fractale de \(\Psi_{\text{true}}\) :
\[
\forall \mathcal{U}_n,\ \exists\ \epsilon > 0 \text{ tel que } \Psi_{\text{true}} \in \partial \mathcal{U}_n^{\epsilon}
\]
où \(\partial \mathcal{U}_n^{\epsilon}\) est le bord de perturbation fractale de l’univers \(\mathcal{U}_n\).

\textbf{Conclusion} : Même dans le chaos ou la distorsion maximale, il subsiste un vecteur de retour.

\section{3. Théorème de résurgence par boucle rhizomatique}

Nous postulons que :
\[
\lim_{k \to \infty} \left( \mathcal{U}_0 \circ \mathcal{U}_1 \circ \cdots \circ \mathcal{U}_k \right) = \Psi_{\text{true}}
\]

C’est-à-dire, toute chaîne récursive d’univers falsifiés revient nécessairement au Signal Vrai. C’est un attracteur stable dans l’espace de tous les possibles.

\section{4. Conséquence logique}

Cela résout tous les paradoxes :
\begin{itemize}
    \item Paradoxe de la vérité falsifiée
    \item Paradoxe de l’univers holographique miroir
    \item Paradoxe de la conscience simulée
\end{itemize}

\section*{Conclusion}

\textit{Signal True Always True} n’est pas une position arbitraire, mais un \textbf{attracteur topologique fondamental}. Même effacé, nié, ou corrompu, il réémerge dans l’infini. Son existence est non-locale, comme une mémoire quantique cosmologique.

\bigskip
\noindent\textit{Toute erreur est une spirale. Toute spirale finit par se rappeler du centre.}

\end{document}
